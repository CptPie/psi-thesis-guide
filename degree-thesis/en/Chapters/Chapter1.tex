% Chapter 1

\chapter{The PSIThesis Template} % Main chapter title

\label{Chapter1} % For referencing the chapter elsewhere, use \ref{Chapter1}

%----------------------------------------------------------------------------------------

% Define some commands to keep the formatting separated from the content
\newcommand{\keyword}[1]{\textbf{#1}}
\newcommand{\tabhead}[1]{\textbf{#1}}
\newcommand{\code}[1]{\texttt{#1}}
\newcommand{\file}[1]{\texttt{#1}}
\newcommand{\option}[1]{\texttt{\itshape#1}}

%----------------------------------------------------------------------------------------

This introductory chapter will give you an overview of the PSI Thesis template and its usage.
Moreover, it presents some pointers for learning \LaTeX{}.

\section{A Quick Welcome}
Welcome to this \LaTeX{} Thesis Template, a beautiful and easy to use template for writing a thesis using the \LaTeX{} typesetting system.

If you are writing a thesis (or will be in the future) and its subject is technical or mathematical (though it doesn't have to be), then creating it in \LaTeX{} is highly recommended as a way to make sure you can just get down to the essential writing without having to worry over formatting or wasting time arguing with your word processor.

\LaTeX{} is easily able to professionally typeset documents that run to hundreds or thousands of pages long. With simple mark-up commands, it automatically sets out the table of contents, margins, page headers and footers and keeps the formatting consistent and beautiful. One of its main strengths is the way it can easily typeset mathematics, even \emph{heavy} mathematics. Even if those equations are the most horribly twisted and most difficult mathematical problems that can only be solved on a super-computer, you can at least count on \LaTeX{} to make them look stunning.
%----------------------------------------------------------------------------------------

\section{Learning \LaTeX{}}

\LaTeX{} is not a \textsc{wysiwyg} (What You See is What You Get) program, unlike word processors such as Microsoft Word or Apple's Pages. Instead, a document written for \LaTeX{} is actually a simple, plain text file that contains \emph{no formatting}. You tell \LaTeX{} how you want the formatting in the finished document by writing in simple commands amongst the text, for example, if I want to use \emph{italic text for emphasis}, I write the \verb|\emph{text}| command and put the text I want in italics in between the curly braces. This means that \LaTeX{} is a \enquote{mark-up} language (like HTML).

\subsection{A (not so short) Introduction to \LaTeX{}}

\marginnote{This template has wide margins. You can liberally use margin notes in your text.}
If you are new to \LaTeX{}, there is a good eBook -- freely available online as a PDF file -- called, \enquote{The Not So Short Introduction to \LaTeX{}}. The book's title is typically shortened to just \emph{lshort}. You can download the latest version (as it is occasionally updated) from here:
\url{http://www.ctan.org/tex-archive/info/lshort/english/lshort.pdf}

It is recommended to spend time to learn how to use \LaTeX{} by creating small test documents. You can also learn from others by looking at the templates hosted at
\url{http://www.LaTeXTemplates.com}.
Making the effort now means you're not stuck learning the system when what you \emph{really} need to be doing is writing your thesis.

\subsection{A Short Math Guide for \LaTeX{}}

If you are writing a technical or mathematical thesis, then you may want to read the document by the AMS (American Mathematical Society) called, \enquote{A Short Math Guide for \LaTeX{}}. It can be found online at
\url{http://www.ams.org/tex/amslatex.html}
under the \enquote{Additional Documentation} section towards the bottom of the page.

\subsection{Common \LaTeX{} Math Symbols}
There are a multitude of mathematical symbols available for \LaTeX{} and it would take a great effort to learn the commands for them all. The most common ones you are likely to use are shown on this page:
\url{http://www.sunilpatel.co.uk/latex-type/latex-math-symbols/}

You can use this page as a reference or crib sheet, the symbols are rendered as large, high quality images so you can quickly find the \LaTeX{} command for the symbol you need.

\subsection{\LaTeX{} Distributions}

The \LaTeX{} distribution is available for Windows, Linux, and macOS\@.
On Windows and Linux systems, the recommended distribution is \textsc{TexLive}.

The package for macOS is called \textsc{MacTeX} and it contains all the applications you need -- bundled together and pre-customized -- for a fully working \LaTeX{} environment and work flow. \textsc{MacTeX} includes a custom dedicated \LaTeX{} editor called \textsc{TeXShop} for writing your `\file{.tex}' files and \textsc{BibDesk}, a program to manage your references and create your bibliography section.



%----------------------------------------------------------------------------------------

\section{Getting Started with this Template}

Once you are familiar with \LaTeX{}, you should explore the directory structure of the template and then proceed to place your own information into the \emph{THESIS INFORMATION} block of the \file{main.tex} file. You can then modify the rest of this file to your unique specifications based on your degree/university. \Cref{FillingFile} on page~\pageref{FillingFile} will help you do this. Make sure you also read Sect.~\ref{ThesisFeatures} about thesis conventions to get the most out of this template.

If you are new to \LaTeX{} it is recommended that you carry on reading through the rest of the information in this document.

\subsection{About this Template}

The template has its own \emph{document class}, which is defined in \file{PSIThesis.cls} file. The class file has been adapted from \file{MastersDoctoralThesis.cls}, which was obtained from \url{https://www.latextemplates.com/template/masters-doctoral-thesis}.

This \LaTeX{} Thesis Template is originally based and created around a \LaTeX{} style file created by Steve R.\ Gunn from the University of Southampton (UK), department of Electronics and Computer Science. You can find his original thesis style file at his site, here:
\url{http://www.ecs.soton.ac.uk/~srg/softwaretools/document/templates/}

Steve's \file{ecsthesis.cls} was then taken by Sunil Patel who modified it by creating a skeleton framework and folder structure to place the thesis files in. The resulting template can be found on Sunil's site here:
\url{http://www.sunilpatel.co.uk/thesis-template}

Sunil's template was made available through \url{http://www.LaTeXTemplates.com} where it was modified many times based on user requests and questions. Version 2.0 and onwards of this template represents a major modification to Sunil's template and is, in fact, hardly recognisable. The work to make version 2.0 possible was carried out by \href{mailto:vel@latextemplates.com}{Vel} and Johannes Böttcher.

%----------------------------------------------------------------------------------------

\section{What this Template Includes}

\subsection{Folders}

This template comes as a single ZIP file that expands out to several files and folders. The folder names are mostly self-explanatory:

\keyword{Appendices} -- this is the folder where you put the appendices. Each appendix should go into its own separate \file{.tex} file. An example and template are included in the directory.

\keyword{Chapters} -- this is the folder where you put the thesis chapters. A thesis usually has about six chapters, though there is no hard rule on this. Each chapter should go in its own separate \file{.tex} file and they can be split as:
\begin{itemize}
\item Chapter 1: Introduction to the thesis topic
\item Chapter 2: Background information and theory
\item Chapter 3: (Laboratory) experimental setup
\item Chapter 4: Details of experiment 1
\item Chapter 5: Details of experiment 2
\item Chapter 6: Discussion of the experimental results
\item Chapter 7: Conclusion and future directions
\end{itemize}
This chapter layout is specialised for the experimental sciences, your discipline may be different.

\keyword{Figures} -- this folder contains all figures for the thesis. These are the final images that will go into the thesis document.

\subsection{Files}

Included are also several files, most of them are plain text and you can see their contents in a text editor. After initial compilation, you will see that more auxiliary files are created by \LaTeX{} or BibTeX and which you don't need to delete or worry about:

\keyword{example.bib} -- this is an important file that contains all the bibliographic information and references that you will be citing in the thesis for use with BibTeX\@. You can write it manually, but there are reference manager programs available that will create and manage it for you. Bibliographies in \LaTeX{} are a large subject and you may need to read about BibTeX before starting with this. Modern reference managers will allow you to export your references in BibTeX format which greatly eases the amount of work you have to do.

\keyword{PSIThesis.cls} -- this is an important file. It is the class file that tells \LaTeX{} how to format the thesis.

\keyword{main.pdf} -- this is your beautifully typeset thesis (in the PDF file format) created by \LaTeX{}. It is supplied in the PDF with the template and after you compile the template you should get an identical version.

\paragraph{Compiling the PDF}

We have tested the PSIThesis with \texttt{LuaLaTeX}. Using \texttt{XeTeX} may also work, but \texttt{pdflatex} will definitely not work as we rely on TTF and OTF fonts.

For compilation on Linux and macOS, you can make use of the provided \keyword{Makefile}. Just navigate to the ``en'' directory and enter \texttt{make} in a terminal. This will make calls to the programs \texttt{lualatex} (which creates the PDF) and \texttt{biber} (which is used to compile the bibliography). The \texttt{make} command keeps track of changes you make to your source files. If you add additional files that should be tracked for changes, you should edit the list of files at the top of this file. Otherwise, \texttt{make} may refuse to compile a new version because it believes that \file{main.pdf} is already up to date. In this case, a call to \texttt{make clean} will help: It removes all files generated during compilation. After that, a call to \texttt{make} will regenerate them, including \file{main.pdf}.

\keyword{main.tex} -- this is an important file. This is the file that you tell \LaTeX{} to compile to produce your thesis as a PDF file. It contains the framework and constructs that tell \LaTeX{} how to layout the thesis. It is heavily commented so you can read exactly what each line of code does and why it is there. After you put your own information into the \emph{THESIS INFORMATION} block -- you have now started your thesis!

Files that are \emph{not} included, but are created by \LaTeX{} as auxiliary files include:

\keyword{main.aux} -- this is an auxiliary file generated by \LaTeX{}, if it is deleted \LaTeX{} simply regenerates it when you run the main \file{.tex} file.

\keyword{main.bbl} -- this is an auxiliary file generated by BibTeX, if it is deleted, BibTeX simply regenerates it when you run the \file{main.aux} file. Whereas the \file{.bib} file contains all the references you have, this \file{.bbl} file contains the references you have actually cited in the thesis and is used to build the bibliography section of the thesis.

\keyword{main.blg} -- this is an auxiliary file generated by BibTeX, if it is deleted BibTeX simply regenerates it when you run the main \file{.aux} file.

\keyword{main.lof} -- this is an auxiliary file generated by \LaTeX{}, if it is deleted \LaTeX{} simply regenerates it when you run the main \file{.tex} file. It tells \LaTeX{} how to build the \emph{List of Figures} section.

\keyword{main.log} -- this is an auxiliary file generated by \LaTeX{}, if it is deleted \LaTeX{} simply regenerates it when you run the main \file{.tex} file. It contains messages from \LaTeX{}, if you receive errors and warnings from \LaTeX{}, they will be in this \file{.log} file.

\keyword{main.lot} -- this is an auxiliary file generated by \LaTeX{}, if it is deleted \LaTeX{} simply regenerates it when you run the main \file{.tex} file. It tells \LaTeX{} how to build the \emph{List of Tables} section.

\keyword{main.out} -- this is an auxiliary file generated by \LaTeX{}, if it is deleted \LaTeX{} simply regenerates it when you run the main \file{.tex} file.

So from this long list, only the files with the \file{.bib}, \file{.cls} and \file{.tex} extensions are the most important ones. The other auxiliary files can be ignored or deleted as \LaTeX{} and BibTeX will regenerate them.

%----------------------------------------------------------------------------------------

\section{Filling in Your Information in the \file{main.tex} File}\label{FillingFile}

You will need to personalise the thesis template and make it your own by filling in your own information. This is done by editing the \file{main.tex} file in a text editor or your favourite LaTeX environment.

Open the file and scroll down to the third large block titled \emph{THESIS INFORMATION} where you can see the entries for \emph{University Name}, \emph{Department Name}, etc.

Fill out the information about yourself, your group and institution. You can also insert web links, if you do, make sure you use the full URL, including the \code{http://} for this. If you don't want these to be linked, simply remove the \verb|\href{url}{name}| and only leave the name.

When you have done this, save the file and recompile \code{main.tex}. All the information you filled in should now be in the PDF, complete with web links. You can now begin your thesis proper!

%----------------------------------------------------------------------------------------

\section{The \code{main.tex} File Explained}

The \file{main.tex} file contains the structure of the thesis. There are plenty of written comments that explain what pages, sections and formatting the \LaTeX{} code is creating. Each major document element is divided into commented blocks with titles in all capitals to make it obvious what the following bit of code is doing. Initially there seems to be a lot of \LaTeX{} code, but this is all formatting, and it has all been taken care of so you don't have to do it.

\marginnote{Please check whether the text of the Declaration of Authorship is correct.}Begin by checking that your information on the title page is correct. For the thesis declaration, your institution may insist on something different than the text given. If this is the case, just replace what you see with what is required in the \emph{DECLARATION PAGE} block.

\marginnote{This is commented out by default.}Then comes a page which contains a funny quote. You can put your own, or quote your favourite scientist, author, person, and so on. Make sure to put the name of the person who you took the quote from.

Following this is the abstract page which summarises your work in a condensed way.

\marginnote{This is commented out by default.}Next come the acknowledgements. On this page, write about all the people who you wish to thank (not forgetting parents, partners and your advisor/supervisor).

The contents pages, list of figures and tables are all taken care of for you and do not need to be manually created or edited. The next set of pages are more likely to be optional and can be deleted since they are for a more technical thesis: insert a list of abbreviations you have used in the thesis, then a list of the physical constants and numbers you refer to and finally, a list of mathematical symbols used in any formulae. Making the effort to fill these tables means the reader has a one-stop place to refer to instead of searching the internet and references to try and find out what you meant by certain abbreviations or symbols.

The list of symbols is split into the Roman and Greek alphabets. Whereas the abbreviations and symbols ought to be listed in alphabetical order (and this is \emph{not} done automatically for you) the list of physical constants should be grouped into similar themes.

\marginnote{This is commented out by default.}The next page contains a one line dedication. Who will you dedicate your thesis to?

Finally, there is the block where the chapters are included. Uncomment the lines (delete the \code{\%} character) as you write the chapters. Each chapter should be written in its own file and put into the \emph{Chapters} folder and named \file{Chapter1}, \file{Chapter2}, etc\ldots Similarly for the appendices, uncomment the lines as you need them. Each appendix should go into its own file and placed in the \emph{Appendices} folder.

After the preamble, chapters and appendices finally comes the bibliography. The bibliography style (called \option{alpha}) is used for the bibliography and is a fully featured style that will even include links to where the referenced paper can be found online. Of course, this relies on you putting the URL information into the BibTeX file in the first place.

%----------------------------------------------------------------------------------------

\section{Template Features}\label{ThesisFeatures}

\subsection{Printing Format}

\marginnote{At the PSI Chair, we highly encourage you to use double-sided printing.}This thesis template is designed for double-sided printing (i.\,e., content on the front and back of pages) as most theses are printed and bound this way. Switching to one-sided printing is as simple as uncommenting the \option{oneside} option of the \code{documentclass} command at the top of the \file{main.tex} file. You may then wish to adjust the margins to suit specifications from your institution.

The headers for the pages contain the page number on the outer side (so it is easy to flick through to the page you want) and the chapter name on the inner side.

The text is set to 11 point by default with single line spacing,\marginnote{For a thesis at the PSI Chair, stick with the defaults.} again, you can tune the text size and spacing should you want or need to using the options at the top of \file{main.tex}. The spacing can be changed similarly by replacing the \option{singlespacing} with \option{onehalfspacing} or \option{doublespacing}.

\subsection{Using US Letter Paper}

\marginnote{For a thesis at the PSI Chair, stick with the defaults.}
The paper size used in the template is A4, which is the standard size in Europe. If you are using this thesis template elsewhere and particularly in the United States, then you may have to change the A4 paper size to the US Letter size. This can be done in the margins settings section in \file{main.tex}.

Due to the differences in the paper size, the resulting margins may be different to what you like or require (as it is common for institutions to dictate certain margin sizes). If this is the case, then the margin sizes can be tweaked by modifying the values in the same block as where you set the paper size. Now your document should be set up for US Letter paper size with suitable margins.

\subsection{References}

The \code{biblatex} package is used to format the bibliography and inserts references such as this one \parencite{murdoch_steven_j._chip_2010}. The options used in the \file{main.tex} file mean that the in-text citations of references are formatted with the author(s) initials and the year \cite{anderson_ross_emv:_2014} of the publication. Multiple references are separated by semicolons (e.\,g., \cite{solat_security_2017, bond_chip_2014}). This is done automatically for you. To see how you use references, have a look at the \file{Chapter1.tex} source file. Reference managers allow you to simply copy and paste or drag references into the document.

The bibliography is typeset with references listed in alphabetical order by the first author's last name. This is similar to the APA referencing style. To see how \LaTeX{} typesets the bibliography, have a look at the very end of this document (or just click on the reference number links in in-text citations).

\paragraph{A Quick Note on bibtex}

As the ``old'' \code{bibtex} backend does not correctly handle unicode character encoding (i.\,e., ``international'' characters), we use the more modern \code{biber} BibTeX engine in this template.

Let's cite a lot of references so that the list of references gets populated \cite{murdoch_steven_j._chip_2010,anderson_ross_emv:_2014,kou_weidong_secure_2003,solat_security_2017,bond_chip_2014,ortiz_s._is_2006,haselsteiner_security_2006,galloway_visa_2019,zhou_nshield_2014,lalehTaxonomyFraudsFraud2009,ferradiWhenOrganizedCrime2016,emvco_emvco_website,emvco_emvco_book1,emvco_emvco_book2,emvco_emvco_book3,emvco_emvco_book4,Yang10,Kopsell06,VilaGM03,Herrmann12-ipv6prefix,Herrmann14-diss,HBF:2013,Herrmann11-NordSec,AcarEEJND14,Herrmann09,WangG13,Raymond00,Hintz02,Herrmann14-encdns,FederrathFHP11,Goodson12-privacy,Fabian10,Google12,goldberg97,WendolskyHF07,chaum81,BertholdFK00,Dingledine04,rfc5246,LoesingMD10,FuchsHF13}.



%----------------------------------------------------------------------------------------

\section{In Closing}

You have reached the end of this introductory chapter. The following chapter continues with a set of recommendations and conventions.

When you are ready, you can rename or overwrite this PDF file and begin writing your own \file{Chapter1.tex} and the rest of your thesis. The tedious tasks of setting up the structure and framework has been taken care of for you. It's now your job to fill it out.

Good luck and happy writing!

\begin{flushright}
Guide written by ---\\
Sunil Patel: \href{http://www.sunilpatel.co.uk}{www.sunilpatel.co.uk}\\
Vel: \href{http://www.LaTeXTemplates.com}{LaTeXTemplates.com}\\[2ex]
with amendments by Dominik Herrmann
\end{flushright}