% Appendix A
 
\chapter{More Information About the Template}

\section{Template Features}\label{ThesisFeatures}

\subsection{Printing Format}

\marginnote{At the PSI Chair, we highly encourage you to use double-sided printing.}This thesis template is designed for double-sided printing (i.\,e., content on the front and back of pages) as most theses are printed and bound this way. Switching to one-sided printing is as simple as uncommenting the \option{oneside} option of the \code{documentclass} command at the top of the \file{main.tex} file. You may then wish to adjust the margins to suit specifications from your institution.

The headers for the pages contain the page number on the outer side (so it is easy to flick through to the page you want) and the chapter name on the inner side.

The text is set to 11 point by default with single line spacing,\marginnote{For a thesis at the PSI Chair, stick with the defaults.} again, you can tune the text size and spacing should you want or need to using the options at the top of \file{main.tex}. The spacing can be changed similarly by replacing the \option{singlespacing} with \option{onehalfspacing} or \option{doublespacing}.

\subsection{Using US Letter Paper}

\marginnote{For a thesis at the PSI Chair, stick with the defaults.}
The paper size used in the template is A4, which is the standard size in Europe. If you are using this thesis template elsewhere and particularly in the United States, then you may have to change the A4 paper size to the US Letter size. This can be done in the margins settings section in \file{main.tex}.

Due to the differences in the paper size, the resulting margins may be different to what you like or require (as it is common for institutions to dictate certain margin sizes). If this is the case, then the margin sizes can be tweaked by modifying the values in the same block as where you set the paper size. Now your document should be set up for US Letter paper size with suitable margins.

\subsection{References}

The \code{biblatex} package is used to format the bibliography and inserts references such as this one \parencite{murdoch_steven_j._chip_2010}. The options used in the \file{main.tex} file mean that the in-text citations of references are formatted with the author(s) initials and the year \cite{anderson_ross_emv:_2014} of the publication. Multiple references are separated by semicolons (e.\,g., \cite{solat_security_2017, bond_chip_2014}). This is done automatically for you. To see how you use references, have a look at the \file{Chapter1.tex} source file. Reference managers allow you to simply copy and paste or drag references into the document.

The bibliography is typeset with references listed in alphabetical order by the first author's last name. This is similar to the APA referencing style. To see how LaTeX typesets the bibliography, have a look at the very end of this document (or just click on the reference number links in in-text citations).

\paragraph{BibTeX Backend}

As the ``old'' \code{bibtex} backend does not correctly handle unicode character encoding (i.\,e., ``international'' characters), we use the more modern \code{biber} BibTeX engine in this template.

Here, we cite a lot of references so that the list of references gets populated \cite{murdoch_steven_j._chip_2010,anderson_ross_emv:_2014,kou_weidong_secure_2003,solat_security_2017,bond_chip_2014,ortiz_s._is_2006,haselsteiner_security_2006,galloway_visa_2019,zhou_nshield_2014,lalehTaxonomyFraudsFraud2009,ferradiWhenOrganizedCrime2016,emvco_emvco_website,emvco_emvco_book1,emvco_emvco_book2,emvco_emvco_book3,emvco_emvco_book4,Yang10,Kopsell06,VilaGM03,Herrmann12-ipv6prefix,Herrmann14-diss,HBF:2013,Herrmann11-NordSec,AcarEEJND14,Herrmann09,WangG13,Raymond00,Hintz02,Herrmann14-encdns,FederrathFHP11,Goodson12-privacy,Fabian10,Google12,goldberg97,WendolskyHF07,chaum81,BertholdFK00,Dingledine04,rfc5246,LoesingMD10,FuchsHF13}.

\section{Listings}

Code listings are done with the \emph{listings} package.

An example can be found in Listing~\ref{lst:listing}.

\begin{lstfloat}
\caption{\label{lst:listing} This is an example of syntax highlighting of Python code with a relatively long caption.}
\begin{lstlisting}[language=Python]
import numpy as np
 
def incmatrix(genl1,genl2):
    m = len(genl1)
    n = len(genl2)
    M = None #to become the incidence matrix
    VT = np.zeros((n*m,1), int)  #dummy variable
 
    #compute the bitwise xor matrix
    M1 = bitxormatrix(genl1)
    M2 = np.triu(bitxormatrix(genl2),1) 
 
    for i in range(m-1):
        for j in range(i+1, m):
            [r,c] = np.where(M2 == M1[i,j])
            for k in range(len(r)):
                VT[(i)*n + r[k]] = 1;
                VT[(i)*n + c[k]] = 1;
                VT[(j)*n + r[k]] = 1;
                VT[(j)*n + c[k]] = 1;
 
                if M is None:
                    M = np.copy(VT)
                else:
                    M = np.concatenate((M, VT), 1)
 
                VT = np.zeros((n*m,1), int)
 
    return M
\end{lstlisting}
\end{lstfloat}

\section{History of this Template}
The template has been adapted from \file{MastersDoctoralThesis.cls}, which was obtained from \url{https://www.latextemplates.com/template/masters-doctoral-thesis}. The template has its own \emph{document class}, which is defined in \file{PSIThesis.cls} file.

The MastersDoctoralThesis LaTeX thesis template is originally based and created around a LaTeX style file created by Steve R.\ Gunn from the University of Southampton (UK), department of Electronics and Computer Science. You can find his original thesis style file at his site, here:
\url{http://www.ecs.soton.ac.uk/~srg/softwaretools/document/templates/}

Steve's \file{ecsthesis.cls} was then taken by Sunil Patel who modified it by creating a skeleton framework and folder structure to place the thesis files in. The resulting template can be found on Sunil's site here:
\url{http://www.sunilpatel.co.uk/thesis-template}

Sunil's template was made available through \url{http://www.LaTeXTemplates.com} where it was modified many times based on user requests and questions. Version 2.0 and onwards of this template represents a major modification to Sunil's template and is, in fact, hardly recognisable. The work to make version 2.0 possible was carried out by \href{mailto:vel@latextemplates.com}{Vel} and Johannes Böttcher.
